\chapter{Tytu� dodatku}
Zasady przyznawania stopnia naukowego doktora i doktora habilitowanego w Polsce okre�la ustawa z dnia 14 marca 2003 r. o stopniach naukowych i~tytule naukowym oraz o stopniach i~tytule w zakresie sztuki (Dz.U. nr 65 z 2003 r., poz. 595 (Dz. U. z 2003 r. Nr 65, poz. 595). Poprzednie polskie uregulowania nie wymaga�y bezwzgl�dnie posiadania przez kandydata tytu�u zawodowego magistra lub r�wnorz�dnego (cho� zasada ta zazwyczaj by�a przestrzegana) i zdarza�y si� nadzwyczajne przypadki nadawania stopnia naukowego doktora osobom bez studi�w wy�szych, np. s�ynnemu matematykowi lwowskiemu � p�niejszemu profesorowi Stefanowi Banachowi. 

W innych krajach r�wnie� zazwyczaj do przyznania stopnia naukowego doktora potrzebny jest dyplom uko�czenia uczelni wy�szej, ale nie wsz�dzie.

